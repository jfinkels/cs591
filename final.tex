%% final.tex - final exam solutions for CS 591 Spring 2015
%%
%% Copyright 2015 Jeffrey Finkelstein.
%%
%% This LaTeX markup document is made available under the terms of the Creative
%% Commons Attribution-ShareAlike 4.0 International License,
%% https://creativecommons.org/licenses/by-sa/4.0/.
\documentclass{article}

\usepackage{amsmath}
\usepackage{amssymb}
%% This must come before hyperref.
\usepackage{amsthm}
%% This is strongly recommended by biblatex.
\usepackage[english]{babel}
\usepackage[backend=biber]{biblatex}
\usepackage[T1]{fontenc}
%% This must come before csquotes.
\usepackage[utf8]{inputenc}
\usepackage{lmodern}
%% This is strongly recommended by biblatex.
\usepackage{csquotes}
%% This must come before hyperref.
\usepackage{thmtools}
%% This must come before complexity.
\usepackage{hyperref}
\usepackage{complexity}
\usepackage{microtype}

%% Set the amount by which certain characters protrude into the margins.
%%
%% \LoadMicrotypeFile{cmr}
%%
%%     This command forces the built-in protrusion settings for the Computer
%%     Modern Roman (cmr) font family to become available at this point, so
%%     that we can override these settings on the next line. Even though we are
%%     really using the Latin Modern Roman (lmr) fonts, microtype uses the cmr
%%     configuration file.
%%
%% \SetProtrusion
%%
%%     This instructs the microtype package that we are going to modify the
%%     protrusion settings.
%%
%% [load=lmr-T1]
%%
%%     Loads the Type 1 (T1) encoding of the lmr font family, thereby setting
%%     the default protrusion values for all the characters. This is only
%%     possible after the \LoadMicrotypeFile{cmr} command (microtype
%%     essentially considers lmr to be an alias for cmr).
%%
%% {encoding=T1, family=lmr}
%%
%%     Indicates that we are going to modify the protrusion values for the T1
%%     encoding of the lmr font family.
%%
%% \textquotedblright = {,1000} (and similar commands)
%%
%%     Force the character given by \textquotedblright to have default
%%     protrusion on the left margin (given by an empty string before the
%%     comma) and full protrusion (that is, protrusion value 1000) on the right
%%     margin.
\LoadMicrotypeFile{cmr}
\SetProtrusion
    [load=lmr-T1]
    {encoding=T1, family=lmr}
    {
      \textquotedblright = {,1000},
      \textquotedblleft = {1000,},
      {'} = {,1000},
      {,} = {,1000},
      {:} = {,1000},
      {;} = {,1000},
      {.} = {,1000}
    }

%% Set the title and author of the PDF file.
\hypersetup{pdftitle={CS 591 Final exam solutions}, pdfauthor={Jeffrey Finkelstein}}

%% Declare the bibliography file.
\addbibresource{final.bib}

%% Declare theorem-like environments.
\declaretheorem[numberwithin=section]{theorem}

%% Custom commands are declared here.
%\newcommand{\email}[1]{\textlangle\href{mailto:#1}{\nolinkurl{#1}}\textrangle}
\newcommand{\todo}[1]{\textbf{TODO #1}}
\newcommand{\CUT}{\mathrm{CUT}}
\newcommand{\LR}{\mathrm{LR}}
\newcommand{\0}{\mathbf{0}}

%% Define the author, title, and date for the document.
\author{Jeffrey~Finkelstein}
\title{CS 591 Final exam solutions}

\begin{document}

\maketitle

%% Document content goes here.

\begin{enumerate}
\item ???
\item
  \begin{enumerate}
  \item Let $K$ denote the complete graph on the set of vertices $V$.
    \begin{align*}
      E_K(S, \bar{S}) & = \left\{ \{u, v\} \in E_K \, \middle| \, u \in S \text{ and } v \in \bar{S}\right\} \\
      & = \left\{ \{u, v\} \in \binom{V}{2} \, \middle| \, u \in S \text{ and } v \in \bar{S}\right\} \\
      & = S \times \bar{S},
    \end{align*}
    where the first equality is by definition, the second is because $E_K = \binom{V}{2}$, and the third because $S$ and $\bar{S}$ form a partition of $V$.
    Thus $|E_K(S, \bar{S})| = |S \times \bar{S}| = |S| |\bar{S}|$.
    We conclude that
    \begin{equation*}
      \min_{S \subseteq V} \frac{|E(S, \bar{S})|}{|S| |\bar{S}|}
      =
      \min_{S \subseteq V} \frac{|E(S, \bar{S})|}{|E_K(S, \bar{S})|}
      =
      \min_{S \subseteq V} \frac{|E(S, \bar{S})|}{|E_H(S, \bar{S})|}.
    \end{equation*}
  \item (I consulted Luca's lecture notes in addition to yours for this problem.)
    We know by the Leighton--Rao Theorem that $\LR_G \leq \alpha_G \leq O(\log n) \LR_G$.
    Define $\alpha_{G, H}$ and $\LR_{G, H}$ by
    \begin{equation*}
      \alpha_{G, H} = \min_{\delta \in \CUT_V} \frac{\delta(G)}{\delta(H)}
      \qquad \text{and} \qquad
      \LR_{G, H} = \min_{\delta \in M_V} \frac{\delta(G)}{\delta(H)},
    \end{equation*}
    where $\CUT_V$ is the cone of cut metrics on vertex set $V$ and $M_V$ is the set of all metrics on $V$.
    A proof similar to that of the Leighton--Rao Theorem shows that $\LR_{G, H} \leq \alpha_{G, H} \leq O(\log k) \LR_{G, H}$.
    Then the algorithm for finding an $O(\log k)$-approximate $H$-sparsest cut is
    \begin{enumerate}
    \item Compute $\delta^*$, the minimum over all metrics $\delta$ of the ratio $\frac{\delta(G)}{\delta(H)}$.
    \item Use Bourgain's Theorem to compute an $\ell_1$-embedding of $\delta^*$ with $O(\log k)$ distortion.
    \item Use the algorithm implicit in Cheeger's inequality to compute a cut from the $\ell_1$-embedding.
    \end{enumerate}
    Computing the optimum $\delta^*$, computing a low distortion $\ell_1$-embedding from a metric, and finding a cut from an $\ell_1$-embedding can all be done in probabilistic polynomial time.
  \end{enumerate}
\item
  \begin{enumerate}
  \item
    We assume the domain of the optimization problems is the set of non-zero real vectors.
    First we put the original optimization problem into the equivalent minimization form,
    \begin{align*}
      \text{minimize } & -x^T A x \\
      \text{subject to } & x^T x - 1 = 0.
    \end{align*}
    Using the notation from class, we let $f(x) = -x^T A x$ and $h(x) = x^T x - 1$.
    The Lagrangian is defined by $L(x, v) = f(x) + v h(x)$ and the Lagrangian dual function is defined by $g(v) = \min_x L(x, v)$.
    Thus the Lagrangian dual optimization problem is
    \begin{align*}
      \text{maximize } & \min_{x} \left(-x^T A x + v \left(x^T x - 1\right)\right) \\
      \text{subject to } & v \geq 0.
    \end{align*}
    We find the minimum in the objective function of this optimization problem by considering values for $x$ that make the derivative zero.
    The derivative of $-x^T A x + v \left(x^T x - 1\right)$ with respect to $x$ is $-2 A x + 2 v x$.
    This is zero exactly when $Ax = vx$ (excluding the case when $x = \0$, since that is not in the domain), or in other words, when $x$ is a $v$-eigenvector of $A$.
    If $x$ is a $v$-eigenvector of $A$, then $-x^T A x + v \left(x^T x - 1\right) = -v$.
    Thus our Lagrangian dual optimization problem is
    \begin{align*}
      \text{maximize } & -v \\
      \text{subject to } & v \geq 0.
    \end{align*}
  \item
    Since the primal objective function is not convex (because, for example, the Hessian is not necessarily positive semidefinite for an arbitrary matrix $A$), we cannot use Slater's condition to prove strong duality (if strong duality holds).
    Thus we will instead try to determine the duality gap.

    Since $v \mapsto -v$ is a monotonically decreasing function with domain bounded on the left by zero, it achieves its maximum at zero, and that value is zero as well.
    We will show that the optimal value of the primal problem is not zero.

    Since the primal problem has the constraint $x^T x = 1$, we can consider the equivalent problem
    \begin{align*}
      \text{maximize } & \frac{x^T A x}{x^T x} \\
      \text{subject to } & x^T x = 1.
    \end{align*}
    Assuming $A$ is real, the value
    \begin{equation*}
      \max_{x \neq \0} \frac{x^T A x}{x^T x}
    \end{equation*}
    is the Rayleigh quotient for the largest eigenvalue of $A$.
    Therefore the maximum value in the primal problem is the largest eigenvalue of $A$.
    Assuming $A$ is non-zero, if the largest eigenvalue of $A$ were zero, then all eigenvalues of $A$ would be zero, a contradiction with the assumption.
    Thus the largest eigenvalue of $A$ is not zero, and hence the maximum value of the primal problem is not zero.

    We have shown that there is a non-zero duality gap, so strong duality does not hold.
  \end{enumerate}
\item ???
\item ???
\end{enumerate}

%% Print the bibliography section here.
\printbibliography

\end{document}
