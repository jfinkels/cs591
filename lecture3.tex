\documentclass[11pt]{article}
\usepackage{graphicx}
\usepackage{amsmath,amsthm,amsfonts}
\usepackage{epsfig,graphics}
\usepackage{hyperref}
\usepackage{verbatim}

\setlength{\textheight}{8.5in}
\setlength{\evensidemargin}{0.0in}
\setlength{\oddsidemargin}{0.0in}
\setlength{\topmargin}{-0.5in}
\setlength{\textwidth}{6.5in}

\newtheorem{theorem}{Theorem}
\newtheorem*{theorem*}{Theorem}
\newtheorem{claim}{Claim}
\newtheorem*{claim*}{Claim}
\newtheorem{lemma}{Lemma}
\newtheorem*{lemma*}{Lemma}
\newtheorem{exercise}{Exercise}
\newtheorem*{exercise*}{Exercise}
\newtheorem{corollary}{Corollary}
\theoremstyle{definition}
\newtheorem{definition}{Definition}


\newcommand{\handout}[6]{
   \renewcommand{\thepage}{#1-\arabic{page}}
   \noindent
   \begin{center}
      \vbox{
    \hbox to \textwidth { #2 \hfill #3 }
       \vspace{4mm}
       \hbox to \textwidth { {\Large \hfill #4  \hfill} }
       \vspace{2mm}
       \hbox to \textwidth { { #5 \hfill #6} }
      }
   \hrulefill
   \end{center}
   \vspace*{4mm}
}

%\newcommand{\lecture}[5]{\handout{#1}{#2}{#3}{#4}{#5}}


% Types of Variables
\newcommand{\bvar}[1]{\mathbf{#1}} % bold variable
\newcommand{\mvar}[1]{\bvar{#1}} % matrix variable
\newcommand{\vvar}[1]{\vec{#1}} % vector variable

% Domains
\newcommand{\R}{\mathbb{R}}
\newcommand{\Z}{\mathbb{Z}}
\newcommand{\redgevec}{\R^{E}}
\newcommand{\rvertvec}{\R^{V}}
\newcommand{\rPos}{\R^{+}}
\newcommand{\rNonNeg}{\R^{\geq 0}}

% Symbol for definitions
\newcommand{\defeq}{\stackrel{\mathrm{\scriptscriptstyle def}}{=}}

% Optimization
\DeclareMathOperator*{\argmin}{arg\,min}

% Types of Graphs
\newcommand{\nlap}{\mathscr{L}_G}
\newcommand{\lap}{\mvar{L}}
\newcommand{\pseudo}[1]{{#1}^\dagger}
\newcommand{\lapPseudo}{\pseudo{\lap}}
\newcommand{\adj}{A}
\newcommand{\incMatrix}{\mvar{B}}
\newcommand{\diag}{\operatorname{diag}}
\newcommand{\rMatrix}{\mvar{R}} % resistance matrix
\newcommand{\iMatrix}{\mvar{I}} % identity matrix


\usepackage[T1]{fontenc}
\usepackage{lmodern}
\usepackage{microtype}

\newcommand{\1}{\mathbf{1}}
\newcommand{\0}{\mathbf{0}}

\DeclareMathOperator*{\Vol}{Vol}

\begin{document}

\handout{}{CS 591 O1: Iterative Methods for Graph Algorithms and Network Analysis}{Spring 2015}{Lecture 3: Spectral Graph and Graph Partitioning}{Instructor: Lorenzo Orecchia}{Scribe: JF}

% Foreword

% context (focus on anyone) why now? - current situation, and why the need is so important
We have seen that algebraic properties of the adjacency matrix of a graph reveal information about the amount of time required before a random walk on the graph converges to the stationary distribution, a common measure of connectedness for a graph.
% need (focus on readers) why you? - why this is relevant to the reader, and why something needed to be done
However, this approach does not immediately address other measures of connectedness, like how difficult partitioning or clustering a graph may be.
% task (focus on author) why me? - what was undertaken to address the need
We examine these other measures by considering partitions and cuts of a graph.
% object (focus on document) why this document - what the document covers
This lecture describes the conductance of a graph, a measure of how likely it is to escape from any given subset of vertices, as it relates to cuts, partitions, spectral gaps, and the convergence of random walks.

% Summary

% findings (focus on author) what? - what the work revealed when performing the task
We prove the easy part of Cheeger's Inequality, that a large spectral gap implies a large conductance, and begin to provide background information for the hard part of Cheeger's Inequality.
% conclusion (focus on readers) so what? - what the findings mean for the audience
Our goal in proving these (relatively tight) inequalities is to indicate the rough equivalence of conductance, spectral gap, and convergence of random walks.
% perspective (focus on anyone) what now? - what should be done next

\section{Spectral gap approximates conductance of a graph}
% Foreword

% context (focus on anyone) why now? - current situation, and why the need is so important
We know from the previous lecture that the spectral gap of the normalized Laplacian of the graph provides an upper bound on the time required for a random walk to converge to the stationary distribution on the graph: a larger spectral gap means a faster convergence.
% need (focus on readers) why you? - why this is relevant to the reader, and why something needed to be done
Are there other algebraic approximations of rate of convergence, and if so, how difficult are they to compute?
% task (focus on author) why me? - what was undertaken to address the need
We will use conductance of a graph, the smallest likelihood of a random walk escaping any set of vertices in one step.
% object (focus on document) why this document - what the document covers
In this section, we prove the easy part of Cheeger's Inequality,
% Summary
% findings (focus on author) what? - what the work revealed when performing the task
which provides upper and lower bounds on the conductance with respect to the spectral gap.
% conclusion (focus on readers) so what? - what the findings mean for the audience
This relates a geometric interpretation (eigenvalues) to a combinatoric interpretation (partitions) of the connectedness of a graph.
% perspective (focus on anyone) what now? - what should be done next

Suppose $G$ is an undirected graph with vertex set $V$ and edge set $E$.
For any vertex $i$ in $V$, the degree of $i$ is denoted $d_i$.
For any set of vertices $S$, the \emph{cut} induced by $S$, denoted $E(S, \bar{S})$, is defined by
\begin{equation*}
  E(S, \bar{S}) = \left\{\left\{i, j\right\} \in E \,\middle|\, i \in S \text{ and } j \in \bar{S} \right\}
\end{equation*}
The \emph{volume} of $S$, denoted $\Vol(S)$, is defined by
\begin{equation*}
  \Vol(S) = \sum_{i \in S} d_i
\end{equation*}
and the volume of the graph, denoted $\Vol(G)$, equals the volume of the entire vertex set.
In other words, the volume of a set is twice the total number of edges in the subgraph induced by the set.
(Each edge is counted twice, once for each of its endpoints.)

The \emph{conductance} of $S$, denoted $\phi(S)$, is defined by
\begin{equation*}
  \phi(S) = \frac{|E(S, \bar{S})|}{\min(\Vol(S), \Vol(\bar{S}))}
\end{equation*}
and the conductance of the graph $G$, denoted $\phi_G$, is defined by
\begin{equation*}
  \phi_G = \min_{S \subseteq V} \phi(S).
\end{equation*}
In other words, the conductance roughly compares the number of edges in $S$ to the number of edges leaving $S$.
%Aside: This term conductance comes from differential geometry, not from electrical circuits; that type of conductance will appear later.
The conductance of a set is always between zero and one.
If close to zero, there are few edges leaving $S$ and many edges in $S$.
If close to one, there are many edges leaving $S$ and few edges in $S$.

Like the spectral gap (that is, the second largest eigenvalue of the normalized Laplacian of the graph), the conductance of a graph is an estimate of how quickly a random walk converges to the stationary distribution on the edges of the graph.
A large conductance means that every set of vertices has many edges leaving it and therefore a random walk will quickly approach the stationary distribution.
The conductance of a set $S$ is, roughly, the probability that a random walk starting from the uniform distribution on $S$ escapes $S$ within one step.
Under this view, starting from a vertex in $S$, we choose (independently) an edge to step across with probability
\begin{equation*}
  \frac{1}{\min(\Vol(S), \Vol(\bar{S}))},
\end{equation*}
so the probability of leaving $S$ is
\begin{equation*}
  |E(S, \bar{S})| \frac{1}{\min(\Vol(S), \Vol(\bar{S}))},
\end{equation*}
which equals the conductance of the set $S$.

Following this intuition, we can relate the conductance of a graph, $\phi_G$, to the spectral gap of the graph, $\lambda_2$.

\begin{theorem}[Cheeger's Inequality]
  For any undirected graph $G$, we have $\lambda_2 / 2 \leq \phi_G \leq \sqrt{2 \lambda_2}$.
\end{theorem}

In other words, if a graph has a large spectral gap, then the graph has large conductance, and conversely, if a graph has a small spectral gap, then the graph has small conductance.
(Both $\lambda_2$ and $\phi_G$ are real numbers in the interval $[0, 1]$, so $\sqrt{\lambda_2} \geq \lambda_2$.)

The left inequality is easy to prove, and the right hard.
We start with the easy one.

\begin{proof}[Proof of the easy part of Cheeger's Inequality]
  Our goal is to show that $\lambda_2 / 2 \leq \phi_G$, or equivalently, that $\lambda_2 \leq 2 \phi_G$.
  By the Rayleigh quotient corresponding to $\lambda_2$ and by the definition of $\phi_G$, it suffices to show
  \begin{equation*}
    \min_{x \perp D^{\frac{1}{2}} \mathbf{1}} \frac{x^T \mathcal{L} x}{x^T x} \leq 2 \min_{S \subseteq V} \frac{|E(S, \bar{S})|}{\min(\Vol(S), \Vol(\bar{S}))}.
  \end{equation*}
  By the definition of the normalized Laplacian
  \begin{equation*}
    \min_{x \perp D^{\frac{1}{2}} \1} \frac{x^T \mathcal{L} x}{x^T x} = \min_{x \perp D \1} \frac{x^T L x}{x^T D x},
  \end{equation*}
  so it suffices to show
  \begin{equation*}
    \min_{x \perp D \1} \frac{x^T L x}{x^T D x} \leq 2 \min_{S \subseteq V} \frac{|E(S, \bar{S})|}{\min(\Vol(S), \Vol(\bar{S}))}.
  \end{equation*}
  Since we are bounding minimums, it furthermore suffices to show that for each set $S$ there is some vector $y$ orthogonal to $D \1$ such that
  \begin{equation*}
    \frac{y^T L y}{y^T D y} \leq \frac{2 |E(S, \bar{S})|}{\min(\Vol(S), \Vol(\bar{S}))}.
  \end{equation*}
  Let $S$ be an arbitrary subset of the vertices of the graph $G$ and choose $y$ to be the vector defined by
  \begin{equation*}
    y_i =
    \begin{cases}
      \phantom{+}\frac{1}{\Vol(S)} & \text{if } i \in S \\
      -\frac{1}{\Vol(\bar{S})} & \text{if } i \in \bar{S} \\
    \end{cases}
  \end{equation*}
  for each $i$ in $V$.
  This vector is orthogonal to $D \1$ because
  \begin{align*}
    y^T D \1 & = \sum_{i \in V} y_i d_i \\
    & = \sum_{i \in S} \frac{1}{\Vol(S)} d_i - \sum_{i \in \bar{S}} \frac{1}{\Vol(\bar{S})} d_i \\
    & = \frac{1}{\Vol(S)} \sum_{i \in S} d_i - \frac{1}{\Vol(\bar{S})} \sum_{i \in \bar{S}} d_i \\
    & = \frac{1}{\Vol(S)} \Vol(S) - \frac{1}{\Vol(\bar{S})} \Vol(\bar{S}) \\
    & = 1 - 1 \\
    & = 0.
  \end{align*}
  We compute separately the numerator and the denominator of the generalized Rayleigh quotient of $y$.
  For the numerator,
  \begin{align*}
    y^T L y & = \sum_{\{i, j\} \in E} (y_i - y_j)^2 \\
    & = \sum_{\{i, j\} \in E(S, \bar{S})} \left(\frac{1}{\Vol(S)} + \frac{1}{\Vol(\bar{S})}\right)^2 \\
    & = \left(\frac{1}{\Vol(S)} + \frac{1}{\Vol(\bar{S})}\right)^2 \sum_{\{i, j\} \in E(S, \bar{S})} 1 \\
    & = \left(\frac{1}{\Vol(S)} + \frac{1}{\Vol(\bar{S})}\right)^2 |E(S, \bar{S})|.
  \end{align*}
  For the denominator
  \begin{align*}
    y^T D y & = \sum_{i \in V} d_i y_i^2 \\
    & = \sum_{i \in S} d_i \frac{1}{\Vol(S)^2} + \sum_{i \in \bar{S}} d_i \frac{1}{\Vol(\bar{S})^2} \\
    & = \frac{1}{\Vol(S)^2} \sum_{i \in S} d_i + \frac{1}{\Vol(\bar{S})^2} \sum_{i \in \bar{S}} d_i \\
    & = \frac{1}{\Vol(S)^2} \Vol(S) + \frac{1}{\Vol(\bar{S})^2} \Vol(\bar{S}) \\
    & = \frac{1}{\Vol(S)} + \frac{1}{\Vol(\bar{S})}.
  \end{align*}
  Thus
  \begin{align*}
    \frac{y^T L y}{y^T D y} & = |E(S, \bar{S})| \left(\frac{1}{\Vol(S)} + \frac{1}{\Vol(\bar{S})}\right) \\
    & = |E(S, \bar{S})| \left(\frac{\Vol(S) + \Vol(\bar{S})}{\Vol(S) \Vol(\bar{S})}\right) \\
    & = \frac{|E(S, \bar{S})|}{\Vol(S) \Vol(\bar{S})} \Vol(G).
  \end{align*}
  Now it suffices to show that
  \begin{equation*}
    \frac{|E(S, \bar{S})|}{\Vol(S) \Vol(\bar{S})} \Vol(G) \leq \frac{2 |E(S, \bar{S})|}{\min(\Vol(S), \Vol(\bar{S}))},
  \end{equation*}
  or more simply, that
  \begin{equation*}
    \frac{\Vol(G)}{\Vol(S) \Vol(\bar{S})} \leq \frac{2}{\min(\Vol(S), \Vol(\bar{S}))}.
  \end{equation*}
  By an averaging argument, the larger of $\Vol(S)$ and $\Vol(\bar{S})$ is at least half of $\Vol(G)$, so
  \begin{align*}
    \frac{\Vol(G)}{\Vol(S) \Vol(\bar{S})} & = \frac{\Vol(G)}{\min(\Vol(S), \Vol(\bar{S})) \max(\Vol(S), \Vol(\bar{S}))} \\
    & \leq \frac{2 \Vol(G)}{\min(\Vol(S), \Vol(\bar{S})) \Vol(G)} \\
    & = \frac{2}{\min(\Vol(S), \Vol(\bar{S}))},
  \end{align*}
  which is the inequality we sought to prove.
  Since this concludes a sequence of sufficient conditions for showing that $\lambda_2 / 2 \leq \phi_G$, we are finished.
\end{proof}

\section[Relaxation of the cut polytope yields better conductance approximation]{Relaxation of the cut polytope \\ yields better conductance approximation}

\textbf{TODO I'm not sure what the end goal of this section was, so the title and section introduction may not be accurate.}

% context (focus on anyone) why now? - current situation, and why the need is so important
Now that we have a lower bound, we wish to prove an upper bound on the conductance of the graph.
% need (focus on readers) why you? - why this is relevant to the reader, and why something needed to be done
Such an upper bound extends the results of the previous section, which states that a large spectral gap implies a large conductance, by providing the converse: a large conductance implies a large spectral gap.
% task (focus on author) why me? - what was undertaken to address the need
In order to do this, we view the minimization problems corresponding to finding the spectral gap as a relaxation of the minimization problem corrseponding to finding the conductance.
% object (focus on document) why this document - what the document covers
In this section, we start to prove the relationship between $\ell_1$-embeddable metrics and cuts of a graph.

% Summary
% findings (focus on author) what? - what the work revealed when performing the task
% conclusion (focus on readers) so what? - what the findings mean for the audience
% perspective (focus on anyone) what now? - what should be done next

What is the complexity of computing the conductance or the spectral gap of a
graph?  Computing the eigenvalues of a graph is easy, whereas even
approximating the conductance of a graph is hard. \textbf{TODO citation
  needed.}  One way to see this is to consider an inequality implicit in the
proof of the easy part of Cheeger's Inequality (ignoring the multiplicative
constants):
\begin{equation*}
  \lambda_2 = \min_{x \perp ???} \frac{x^T L x}{x^T D x} \leq \min_{x \perp ???} \frac{x^T L x}{x^T D x} \leq \phi_G.
\end{equation*}
The right Rayleigh quotient is minimized over the ``cut vectors'', whereas the left Rayleigh quotient is minimized over \emph{all} vectors orthogonal to ???.
The right minimization has a smaller domain, so it is more difficult to find a minimum.
\textbf{TODO what is a cut vector? Is it a vector of the form $\alpha\1_S + \beta\1_{\bar{S}}$, where $\alpha \neq \beta$?}
Specifically, the right minimization is over the \emph{cut polytope}, the convex hull of cut vectors (\textbf{TODO explain why}), and the left minimization is over the set of all vectors orthogonal to ???.
Expanding the domain of a minimization is called \emph{relaxation}, and can be used in many settings to approximate a minimization over a more restrictive domain.

There is a domain between the cut polytope and the set of all vectors orthogonal to ???: the set of $\ell_1$-embedabble cut metrics.
A \emph{metric on a graph} is an $n \times n$ matrix $d$ over the non-negative reals (thought of as a function from $V^2$ to $\mathbb{R}^+_0$) satisfying
\begin{enumerate}
\item $d_{ii} = 0$ for each $i \in V$,
\item $d_{ij} = d_{ji}$ for each $i, j \in V$ (symmetry),
\item $d_{ij} \leq d_{ik} + d_{kj}$ for each $i, j, k \in V$ (triangle inequality).
\end{enumerate}
For any subset of vertices $S$, the \emph{cut metric for $S$} is the metric defined by
\begin{equation*}
  d_{ij} =
  \begin{cases}
    1 & \text{if } (i \in S \text{ and } j \in \bar{S}) \text{ or } (i \in \bar{S} \text{ and } j \in S) \\
    0 & \text{otherwise,}
  \end{cases}
\end{equation*}
for each $i, j \in V$, or more succinctly, $d = \1_S \1_{\bar{S}}^T + \1_{\bar{S}} \1_S^T$, where $\1_S$ denotes the characteristic vector of $S$ (with a 1 in entry $i$ exactly when $i \in S$).
A metric $d$ is \emph{$\ell_1$-embeddable} if there is a positive integer $k$ and a function $f \colon V \to \mathbb{R}^k$ such that $d_{ij} = \| f(i) - f(j) \|_1$.

%% The quadratic form for the Laplacian of a graph does not induce a metric.
\textbf{TODO Need to explain here what is the end goal of comparing cut metrics and l1 embeddable metrics.}

\begin{lemma}
  A cut metric is an $\ell_1$-embeddable metric.
\end{lemma}
\begin{proof}
  Let $S$ be a subset of vertices and $d$ be its cut metric.
  If we choose $k$ to be 1 and $f$ to be the characteristic function of $S$, then
  \begin{align*}
    d_{i, j} & = (\1_S \1^T_{\bar{S}} + \1_{\bar{S}} \1_S^T)_{ij} \\
    & = (\1_S \1^T_{\bar{S}})_{ij} + (\1_{\bar{S}} \1_S^T)_{ij} \\
    & = (\1_S)_i (\1_{\bar{S}})_j + (\1_{\bar{S}})_i (\1_S)_j \\
    & = f(i) (1 - f(j)) + (1- f(i)) (f_j) \\
    & = f(i) + f(j) - 2 f(i) f(j).
  \end{align*}
  Enumerating $f(i) + f(j) - 2 f(i) f(j)$ and $\|f(i) - f(j)\|_1$ over the four possible pairs of values of $f(i)$ and $f(j)$ reveals that the two expressions are equal.
  Therefore, $d$ is $\ell_1$-embeddable.
\end{proof}

\begin{theorem}
  The convex hull of $\ell_1$-embeddable metrics equals the convex hull of cut metrics.
\end{theorem}
\begin{proof}
  By the previous lemma, each cut metric is and $\ell_1$-embeddable metric, and hence, trivially, a convex combination of $\ell_1$-embeddable metrics.
  Thus it suffices to prove that each $\ell_1$-embeddable metric can be expressed as a convex combination of cut metrics.

  Suppose $d$ is an $\ell_1$-embeddable metric, so there is some positive integer $k$ and function $f \colon V \to \mathbb{R}^k$ such that $d_{ij} = \|f(i) - f(j)\|_1$.
  We wish to express $d$ as $\sum_{S \subseteq V} \alpha_S d^S$, where each $d^S$ is the cut metric for $S$.

  For each dimension $m$ in $\{1, \dotsc, k\}$, the embedding $f$ induces a permutation $\pi$ on the vertices such that $f(\pi(1))_m \leq \dotsb \leq f(\pi(n))_m$.
  In other words, each of the elements $f(i)$ in $\mathbb{R}^k$ can be sorted in nondecreasing order according to the projection onto their $m$th component.
  For each dimension $m$ and each vertex $i$, let $\alpha_{S^m_i} = |f(\pi(i))_m - f(\pi(i + 1))_m|$ and $S^m_i = \{ j \in V \,|\, f(\pi(j)) \leq f(\pi(i)) \}$.
  In other words, $S^m_i$ is the set of all vertices whose images appear before the image of $i$ in the ordering and $\alpha_{S^m_i}$ is the distance from the image of $i$ to the image with the next largest $m$th component.

  Now if we choose all other cuts $S$ to have coefficient $\alpha_S$ equal to zero, then we need only prove that
  \begin{equation*}
    d = \sum_{m = 1}^k \sum_{i = 1}^n \alpha_{S^m_i} d^{S^m_i}.
  \end{equation*}
  Let $u$ and $v$ be vertices in $V$.

  \textbf{TODO I'm not sure how to complete this proof.}
\end{proof}

\end{document}
